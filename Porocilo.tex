\documentclass[12pt,a4paper]{amsart}
% ukazi za delo s slovenscino -- izberi kodiranje, ki ti ustreza
\usepackage[slovene]{babel}
\usepackage[utf8]{inputenc}
\usepackage[T1]{fontenc}
\usepackage{multirow}
\usepackage{amsmath,amssymb,amsfonts}
\usepackage{url}
\usepackage[normalem]{ulem}
\usepackage[dvipsnames,usenames]{color}
\usepackage{csquotes}
\usepackage{caption}
\usepackage{lipsum}
\usepackage{hyperref}
\usepackage{tikz}
\usepackage{listings}
\usepackage{xcolor}
\usepackage{graphicx}
\usepackage{subcaption}

\lstset{
    breaklines=true,    
    breakatwhitespace=false,
    postbreak=\space,   
    tabsize=2,    
    basicstyle=\small\ttfamily\bfseries,
    commentstyle=\color{green!50!black},
    keywordstyle=\color{blue},
    numberstyle=\tiny\color{gray},
    numbers=left
}
\usetikzlibrary{graphs}
\usetikzlibrary{graphs.standard}

\makeatletter
\renewcommand\section{\@startsection{section}{1}%
  \z@{.5\linespacing\@plus.7\linespacing}{.5\linespacing}%
  {\normalfont\scshape\large\centering}}
\renewcommand\subsection{\@startsection{subsection}{2}%
  \z@{.5\linespacing\@plus.7\linespacing}{.5\linespacing}%
  {\normalfont\scshape}}
\renewcommand\subsubsection{\@startsection{subsubsection}{3}%
  \z@{.5\linespacing\@plus.7\linespacing}{-.5em}%
  {\normalfont\itshape}}
\makeatother

% ne spreminjaj podatkov, ki vplivajo na obliko strani
\textwidth 15cm
\textheight 24cm
\oddsidemargin.5cm
\evensidemargin.5cm
\topmargin-5mm
\addtolength{\footskip}{10pt}
\pagestyle{plain}
\overfullrule=15pt % oznaci predlogo vrstico

% ukazi za matematicna okolja
\theoremstyle{plain} % tekst napisan pokoncno
\newtheorem{definicija}{Definicija}[section]
\newtheorem{primer}[definicija]{Primer}
\newtheorem{definition}{Definicija}[section]

\renewcommand\endprimer{\hfill$\diamondsuit$}

% za stevilske mnozice uporabi naslednje simbole
\newcommand{\R}{\mathbb R}
\newcommand{\N}{\mathbb N}
\newcommand{\Z}{\mathbb Z}
\newcommand{\C}{\mathbb C}
\newcommand{\Q}{\mathbb Q}

% ukaz za slovarsko geslo
\newlength{\odstavek}
\setlength{\odstavek}{\parindent}
\newcommand{\geslo}[2]{\noindent\textbf{#1}\hspace*{3mm}\hangindent=\parindent\hangafter=1 #2}

% naslednje ukaze ustrezno popravi
\newcommand{\program}{Finančna matematika} % ime studijskega programa: Matematika/Finančna matematika
\newcommand{\imeavtorja}{Blaž Povh, Luka Houška} % ime avtorja
\newcommand{\imementorja}{doc. dr. Janoš Vidali} % akademski naziv in ime mentorja
\newcommand{\imesomentorja}{prof. dr. Riste Škrekovski}
\newcommand{\naslovdela}{Dominant Edge Metric Dimension}
\newcommand{\letnica}{2024} %letnica 

%%%%%%%%%%%%%%%%%%%%%%%%%%%%%%%%%%%%%%%%%%%%%%%%%%%%%%%%%%%%%%%%%%%%%%%%%%%%%%%%%%%%%%%%%%
\begin{document}

\thispagestyle{empty}
\noindent{\large
UNIVERZA V LJUBLJANI\\[1mm]
FAKULTETA ZA MATEMATIKO IN FIZIKO\\[5mm]
\program\ }
\vfill

\begin{center}{\large
\imeavtorja\\[2mm]
{\bf \naslovdela}\\[10mm]
Skupinski projekt\\[2mm]
Poročilo\\[1cm]
Mentorja: \imementorja, \\ \imesomentorja\\[2mm]}
\end{center}
\vfill

\noindent{\large
Ljubljana, januar \letnica}
\pagebreak

%%%%%%%%%%%%%%%%%%%%%%%%%%%%%%%%%%%%%%%%%%%%%%%%%%%%%%%%%%%%%%%%%
\section{Navodilo naloge}
    Implement an ILP for computing dominant edge metric dimension and answer the following questions.
    \begin{enumerate}
        \item Find trees on $n$ vertices which have minimum/maximum dominant edge metric dimension.
        \item Find trees $T$ on $n$ vertices for which $Dedim(T) - edim(T)$ is maximum/minimum possible.
        \item Find trees $T$ on $n$ vertices for which $Dedim(T)/ edim(T)$ is maximum/minimum possible.
        \item Determine $Dedim(G)$ of a grid graph $Pk \Box Pt$.
    \end{enumerate}

    For small graphs, apply a systematic search; for larger ones, apply some stochastic search. Report your results.
\bigskip

\section{Načrt dela}
    Najina projektna naloga se nanaša dominantne metrične dimenzije grafov. Projekt sva razdelila na več manjših delov.
    Slednje sva si potem enakomerno razdelila. Natančnejši pregled najinega dela je opisan v nadaljevanju, tukaj pa je le 
    kratek opis postopkov najinega dela.
    Najprej sva napisala ustrezen CLP, ki izračuna dominantno metrično dimenzijo povezav za dani graf. V nadaljevanju sva 
    za vsako podnalogo ustvarila nove funkcije, ki so za izbran graf izračunale v navodilih podane karakteristike in generirala
    različne grafe, ter na njih testirala prej napisano kodo. 
    V nadaljevanju sva napisala kodo, ki sva jo testirala tudi na večjih grafih. Temeljila pa je na metahevrističnem
    algoritmu imenovanem \emph{local search}. 
    Najin cilj oz. motivacija je bila ugotoviti, ali za grafe, ki ustrezajo določenim pogojem, veljajo kakšne posebne lastnosti.
    Za majhne grafe, natančneje drevesa, sva se problema lotila s testiranjem na vseh različnih drevesih. Za večje grafe
    pa sva kodo algoritem testirala le na naključnih grafih, saj je ta algoritem časovno zelo potraten. 
    Za reševanje najine naloge sva uporabila okolje Sage (SageMath) znotraj spletne platforme CoCalc. V nadaljevanju bova razložila,
    kako sva se lotila vsake podnaloge, ter dodala tudi kodo s komentarji, ki se nahaja na najinem GitHub repozitoriju.
\bigskip
%%%%%%%%%%%%%%%%%%%%%%%%%%%%%%%%%%%%%%%%%%%%%%%%%%%%%%%%%%%%%%%%%
    
\section{Definicije}
    Za lažje razumevanje najinega problema, si najprej poglejmo par pojmov ter definicij, ki sva jih uporabljala v sklopu najine
    projekne naloge.
    \begin{definicija}
        V grafu $G$, množici vozlišč $S$ pravimo, da razrešuje povezave, če za vsak par povezav $e, f \in E(G)$ obstaja $s \in S$, tako da $d(s, e) \neq  d(s, f)$.
    \end{definicija}

    \begin{definicija}
        Metrična dimenzija povezav na  povezanem grafu $G$, označili jo bomo z $edim(G)$, je moč najmanjše množice vozlišč $S\subseteq V(G)$, ki razlikuje vse pare povezav, t.j. za vsak par povezav $e, f \in E(G)$, obstaja vozlišče $s \in S$, tako da $d(s, e) \neq d(s, f )$. Tu upoštevamo, da je za povezavo $e=uv$, $d(s, e) = min\{d(s, u), d(s, v)\}$.
    \end{definicija}
        
    \begin{definicija}
        Množica vozlišč $C$ je \emph{vozliščno pokritje}, če ima vsaka povezava iz grafa $G$ vsaj eno krajišče v $C$.
    \end{definicija}

    \begin{definicija}
        Dominantna množica, ki razrešuje povezave $S$ za graf $G$, je množica $S$, ki je hkrati vozliščno pokritje in razrešuje povezave. Dominantna metrična dimenzija povezav grafa $G$ je moč najmanjše množice $S$, ki razrešuje povezave. Označimo jo z $Dedim(G)$.
    \end{definicija}


%%%%%%%%%%%%%%%%%%%%%%%%%%%%%%%%%%%%%%%%%%%%%%%%%%%%
\section{Majhni grafi}
    Najin prvi korak je bil generiranje funkcije, ki s pomočjo ustreznega ter učinkovitega CLP izračuna dominantno metrično dimenzijo
    povezav za dani graf, čemur krajše rečemo kar Dedim. Prvo sve definirala funkcijo \emph{razdalje\_povezave\_vozlisca(g)},
    ki kot argument sprejme graf $g$ in vrne matriko razdalj med povezami in vozlišči, torej element v vrstici $i$ in stolpcu $j$ predstavlja razdaljo med povezavo $e_{i}$ in vozliščem $v_{j}$. 

\bigskip
\begin{small}
    \begin{lstlisting}
    def razdalje_povezave_vozlisca(g):
        razdalje = distances_all_pairs(g)
        razdalje_povezave = {}

        for i,j in g.edges(labels=False):
            razdalje_povezave[i,j] = {}
            for v in g:
                razdalje_povezave[i,j][v] = min(razdalje[i][v], razdalje[j][v])

        return razdalje_povezave

    \end{lstlisting}
\end{small}

\bigskip

To funkcijo sva potem uporabila v funkciji \emph{CLP\_Dedim(g)}, ki kot argument sprejme graf $g$ in vrne vrednost $Dedim(g)$, ter
vozlišča, ki sestavljajo rešitev.

\bigskip

\begin{small}
    \begin{lstlisting}
    import itertools  
    def CLP_Dedim(g):

        razdalje = razdalje_povezave_vozlisca(g)

        p = MixedIntegerLinearProgram(maximization = False)
        x = p.new_variable(binary = True)
        p.set_objective( sum([x[v] for v in g]) )

        for u,v in g.edges(labels = False):
            p.add_constraint(x[u] + x[v] >= 1)


        edges = list(g.edges(labels=False))
        for (i,j),(u,v) in itertools.combinations(edges, 2):
            vsota = sum(abs(razdalje[(i,j)][k] - razdalje[(u,v)][k])*x[k] for k in g.vertices())
            p.add_constraint(vsota >= 1)


        optimalna_resitev = p.solve()
        vrednosti_za_S = p.get_values(x)

        return optimalna_resitev, vrednosti_za_S

    \end{lstlisting}
\end{small}

\bigskip

\subsection{Prvi del naloge}
Prvi del naloge je od naju zahteval, da najdeva drevesa $T$ na $n$ vozliščih, ki imajo minimalen/maksimalen $Dedim(T)$. V ta namen sva definirala funkciji \emph{drevesa\_min(i,j)} in \emph{drevesa\_max(i,j)}, 
ki kot argumenta sprejmeta števili $i$ in $j$ in za vsako število vozlišč med $i$ in $j$ narišeta drevo $T$, ki ima najmanjšo oz. največjo $Dedim$ in pod njim izpišeta vrednost $Dedim$. Pomagala sva si z vgrajeno funkcijo \emph{TreeIterator}, ki je zelo uporabna za generiranje vseh dreves na $n$ vozliščih.

\subsection{Drugi del naloge}
Drugi del naloge je od naju zahteval, da najdeva drevesa $T$ na $n$ vozliščih, ki imajo minimalno/maksimalno možno razliko $Dedim(T) - edim(T)$. Tu je $edim(T)$ metrična dimenzija grafa na povezavah, torej opustimo pogoj, da more biti iskana množica vozliščno pokritje. Najina prva naloga je bila, da definirava funkcijo \emph{CLP\_edim(g)}, ki nam vrne metrično dimenzijo grafa. Nato sva definirala funkciji \emph{drevesa\_min\_razlika(i,j)} in \emph{drevesa\_max\_razlika(i,j)}, 
ki kot argumenta sprejmeta števili $i$ in $j$ in za vsako število vozlišč med $i$ in $j$ narišeta drevo $T$, ki ima najmanjšo oz. največjo razliko $Dedim(T) - edim(T)$ in pod njim izpišeta vrednost. Tudi tu sva si pomagala z vgrajeno funkcijo \emph{TreeIterator}.

\subsection{Tretji del naloge}
Tretji del naloge je od naju zahteval, da najdeva drevesa $T$ na $n$ vozliščih, ki imajo minimalen/maksimalen možen količnik $Dedim(T)/edim(T)$. V ta namen sva definirala funkciji \emph{drevesa\_min\_ulomek(i,j)} in \emph{drevesa\_max\_ulomek(i,j)}, 
ki kot argumenta sprejmeta števili $i$ in $j$, $i,j \geq 3$ in za vsako število vozlišč med $i$ in $j$ narišeta drevo $T$, ki ima najmanjši oz. največji količnik $Dedim(T)/edim(T)$ in pod njim izpišeta vrednost. Tudi tu sva si pomagala z vgrajeno funkcijo \emph{TreeIterator}.

\subsection{Četrti del naloge}
Četrti del naloge je od naju zahteval, da izračunava $Dedim(g)$ za kartezične produkte dveh poti $P_{k}\square P_{t}$. V ta namen sva definirala funkcijo \emph{grid\_graph(i,j)}, 
ki kot argument sprejme števili $i$ in $j$, $i \leq j$ in za vse kombinacije števila vozlišč med $i$ in $j$ nariše graf $P_{k}\square P_{t}$ in pod njim izpiše vrednost $Dedim(P_{k}\square P_{t})$. 

%%%%%%%%%%%%%%%%%%%%%%%%%%%%%%%%%%%%%%%%%%%%%%%%%%%%
\section{Ugotovitve}

\subsection{Prvi del naloge}

Hitro je bilo vidno, da so drevesa, na $n$ vozliščih, ki imajo maksimalen $Dedim(g)$ zvezde $S_{n}$, njihov $Dedim(S_{n})$ pa je $n-1$. Za drevesa z minimalnim $Dedim(T)$ nisva našla neke oblike grafa, ki bi se do potankosti ohranjala za vse $n$, sva pa ugotovila, da za 
število vozlišč $n=3k, 3k+1,3k+2$ kjer je $k\geq 2$ velja, da je največja stopnja vozlišča v drevesu $\Delta (T) = k+1$, vozlišče s takšno stopnjo je natanko eno, drevo ima $k+1$ listov, vsa ostala vozlišča $v$, ki niso listi pa imajo $deg_{T}(v)=2$. To sva preverila za drevesa z $\left\lvert V(T)\right\rvert \leq 20$ in nimava razloga, da ne bi verjela, da to velja za vse $n$, bi se pa zavoljo matematične korektnosti to spodobilo dokazati, kar pa bi bilo(v primeru, da je sploh možno) za to projektno nalogo preobsežna naloga in niti ni naloga tega projekta. 

\begin{table}[h]
    \begin{center}
      \begin{tabular}{l|c} % <-- Alignments: 1st column left, 2nd middle and 3rd right, with vertical lines in between
        \textbf{število vozlišč} & \textbf{največja stopnja vozlišča} \\
        \hline
        6 & 3 \\
        7 & 3 \\
        8 & 3 \\
        9 & 4 \\
        10 & 4 \\
        11 & 4 \\
        12 & 5 \\
        13 & 5 \\
        14 & 5
      \end{tabular}
    \end{center}
    \caption{Največja stopnja vozlišča v drevesu z min(Dedim(T))}
    \label{tab:tabela1}
\end{table}

\bigskip
\begin{figure}[h]
    \centering
    \includegraphics[width=0.25\textwidth]{graf1.png}
    \includegraphics[width=0.25\textwidth]{graf2.png}
    \includegraphics[width=0.25\textwidth]{graf3.png}
    \includegraphics[width=0.25\textwidth]{graf4.png}
    \includegraphics[width=0.25\textwidth]{graf5.png}
    \includegraphics[width=0.25\textwidth]{graf6.png}
    \caption{Drevesa z $6 \leq \left\lvert V(T)\right\rvert \leq 11$ z minimalnim $Dedim(T)$}
    \label{fig:slika1}
\end{figure}

\subsection{Drugi del naloge}
Ponovno sva hitro ugotovila vzorec, da so drevesa  $T$ na $n$ vozliščih, ki imajo minimalen možen $Dedim(T) - edim(T)$ zvezde $S_{n}$ in da je $Dedim(T) - edim(T) = 1$. 
Za drevesa z maksimalno razliko $Dedim(T) - edim(T)$ nisva našla oblike grafa, ki bi se do potankosti ohranjala za vse $n$, sva pa ugotovila, da za 
število vozlišč $n=2k, 2k+1$ kjer je $k\geq 3$ velja, da je razlika enaka $k-1$. To sva preverila za drevesa $\left\lvert V(T)\right\rvert \leq 20$ in ponovno nimava razloga, da ne bi verjela, da to velja za vse $n$.

\begin{table}[h]
    \begin{center}
      \begin{tabular}{l|c} % <-- Alignments: 1st column left, 2nd middle and 3rd right, with vertical lines in between
        \textbf{število vozlišč} & \textbf{max(Dedim(T) - edim(T))} \\
        \hline
        6 & 2 \\
        7 & 2 \\
        8 & 3 \\
        9 & 3 \\
        10 & 4 \\
        11 & 4 \\
        12 & 5 \\
        13 & 5 \\
      \end{tabular}
    \end{center}
    \caption{Maksimalna razlika Dedim(T) - edim(T)}
    \label{tab:tabela2}
\end{table}

\subsection{Tretji del naloge}
Ugotovila sva, da so drevesa na $n$ vozliščih, ki imajo minimalen količnik \\
$Dedim(T)/edim(T)$ zvezde $S_{n}$, količnik pa bo za $n\geq 3$ enak $\frac{n-1}{n-2} $. 

\begin{table}[h]
    \begin{center}
      \begin{tabular}{l|c} % <-- Alignments: 1st column left, 2nd middle and 3rd right, with vertical lines in between
        \textbf{število vozlišč} & \textbf{min(Dedim(T)/edim(T))} \\
        \hline
        3 & 2 \\
        4 & 1.5 \\
        5 & 1.3333 \\
        6 & 1.25 \\
        7 & 1.2 \\
        8 & 1.6667 \\
        9 & 1.1429 \\
        10 & 1.125 \\
      \end{tabular}
    \end{center}
    \caption{Minimalen količnik Dedim(T)/edim(T)}
    \label{tab:tabela3}
\end{table}

Ugotovila sva, da drevesom na $n$ vozliščih z maksimalnim količnikom \\
$Dedim(T)/edim(T)$ ustrezajo poti $P_{n}$, količnik pa je za $n\geq 4$ enak $\left\lfloor \frac{n}{2} \right\rfloor $.

\begin{table}[h]
    \begin{center}
      \begin{tabular}{l|c} % <-- Alignments: 1st column left, 2nd middle and 3rd right, with vertical lines in between
        \textbf{število vozlišč} & \textbf{min(Dedim(T)/edim(T))} \\
        \hline
        4 & 2 \\
        5 & 2 \\
        6 & 3 \\
        7 & 3 \\
        8 & 4 \\
        9 & 4 \\
        10 & 5 \\
        11 & 5 \\
      \end{tabular}
    \end{center}
    \caption{Maksimalen količnik Dedim(T)/edim(T)}
    \label{tab:tabela4}
\end{table}

\subsection{Četrti del naloge}
Ugotovila sva, da je $Dedim(P_{k}\square P_{t}) = \left\lfloor \frac{k\cdot t}{2} \right\rfloor$, če je $k+t \geq 5$. Odkrila sva tudi katera vozlišča nastopajo v iskani množici. 
Če je $k\cdot t$ liho število, torej če sta $k,t$ lihi potem so izbrana vozlišča (pobarvana z rdečo) kot sledi:

\bigskip
\begin{figure}[h]
    \centering
    \includegraphics[width=0.25\textwidth]{graf7.png}
    \caption{Izbrana vozlišča, če je število vozlišč $P_{k}\square P_{t}$ liho.}
    \label{fig:slika2}
\end{figure}

\bigskip
Če je $k\cdot t$ sodo število, torej, če je vsaj eden izmed $k,t$ sod potem so izbrana vozlišča kot sledi.
\bigskip
\begin{figure}[h]
    \centering
    \includegraphics[width=0.25\textwidth]{graf8.png}
    \caption{Izbrana vozlišča, če je število vozlišč $P_{k}\square P_{t}$ sodo.}
    \label{fig:slika2}
\end{figure}



%%%%%%%%%%%%%%%%%%%%%%%%%%%%%%%%%%%%%%%%%%%%%%%%%%%%%
\section{Local search}
    Metodo local search sva dodala kot alternativo za velike grafe, saj najde približek rešitve dosti hitreje kot 
    računanje prek CLP. Napisani algoritem deluje tako, da za podani graf najprej kreira poljubno dopustno rešitev, 
    ki pa v večini primerov ni optimalna. Torej za začetno rešitev, si izbere poljubno množico vozlišč danega grafa,
    za katero velja, da razrešuje povezave ter je vozliščno pokritje. Ko imamo to začetno rešitev se lotimo iskanja 
    optimalne rešitve. Pogledamo vsako vozlišče v začetni rešitvi in pogledamo, če imamo dopustno rešitev tudi, če 
    začetni rešitvi odstranimo to vozlišče. V primeru, da imamo optimalno rešitev, ko odstranimo vozlišče gremo potem 
    naprej do drugih vozlišč in tako preverimo za vsako vozlišče. To ni sicer najbolj "tipičen" local search algoritem.
    Primerjenje bi sicer bilo, da bi za začetno rešitev na neki okolici pogledali, ali dobimo boljšo rešitev, če poljubno
    vozlišče iz začetne rešitve zamenjamo z kakšnim sosedom v grafu. Vendar za najin problem to ni bilo izvedljivo, saj
    je bilo časovno preveč potratno. Ker je pa slabost najinega algoritma ta, da je končna rešitev močno odvisna od začetne
    rešitve, sva še malce izboljšala algoritem. Torej, za dani graf ustvarimo $k$ začetnih rešitev in za vsako začetno
    rešitev izvedemo potem local search. Za končno rešitev pa potem vzamemo najbolj optimalno od vseh rešitev. Torej
    rešitev, kjer je množica vozlišč v rešitvi najmanjša. Ob testiranju algoritma na poljubnih grafih lahko ugotovimo, 
    da algoritem ni pogosto različen od optimalne rešitve, je pa dosti hitrejši od CLP pri velikih grafih. 
    Glede na preizuse sva ugotovila, da se za grafe do nekje 18 vozlišč splača uporabiti CLP, za večje grafe pa algoritem local search.


%%%%%%%%%%%%%%%%%%%%%%%%%%%%%%%%%%%%%%%%%%%%%%%%%%%%%
\bigskip
Vse funkcije, ki so omenjene v poročilu so napisane na Github-u v datoteki \emph{clp.py}.
Za pomoč pri nalogi sva uporabila spodnjo literaturo.
\begin{thebibliography}{99}
    \bibitem{dominantedge}
    M. Tavakoli, M. Korivand, A. Erfanian, G. Abrishami, E. T. Baskoro,
    \emph{The dominant edge metric dimension of graphs},
    Electronic Journal of Graph Theory and Applications 11(1) (2023) 197–208.  
\end{thebibliography}


\end{document}




















\end{document}